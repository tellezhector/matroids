\prob
{\label{p11}
    Let $r$ be an integer exceeding one and $W_r$ be the $r$-wheel. Show that:
    \begin{enumerate}[label=(\roman*)]
        \item $W_3 \cong K_4$.
        \item $W_r^* \cong W_r$.
        \item $M(W_4)$ is isomorphic to a restriction of $M^*(K_{3,3})$.
    \end{enumerate}
}
\begin{proof}$\,$\pn

    \begin{enumerate}
        \item   It is sufficient to prove that $W_3$ is complete.\pn
                The central vertex is always adjacent to the rest of the vertices.\pn
                Any vertex in the exterior cycle is adjacent to extactly other two of its type, but as $W_3$ has
                only 3 vertices in its exterior cycle, then any vertex in the exterior cycle is adjacent to the other
                two, plus, it is adjacent to the center.\pn
                
                We have proved that any vertex has degree 3 and there are 4 vertices. Then the graph is complete and
                we are done.\pn
                
        \item   Any triangle that is induced by the central vertex and two of the external vertices under the
                dual operator will end up as a vertex of degree tree, adjacent to two triangles of its type and
                to the exterior face.\pn
                
                There are exactly $r$ triangles of this type in $W_r$, then the external face will be adjacent to
                every one of these triangles.\pn
                
                Then the dual operator will make a graph such that it has a vertex adjacent to the rest of the vertices and
                the rest of the vertices are adjacent in addition to exactly two of its type. This is, the dual of 
                $W_r$ is $W_r$ itself.\pn
                
        \item   It is easy to see that $W_4 \cong K_{3, 3} / \{e\}$ for any $e \in E(K_{3, 3})$. Using [\ref{p14}], we have that
                $M^*(K_{3, 3} / \{e\}) \cong M^*(W_4) \cong M(W_4^*) \cong M(W_4)$.
                
    \end{enumerate}
\end{proof}