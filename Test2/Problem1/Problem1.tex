\prob{
    Find each of the following:
    \begin{enumerate}[label=(\roman*)]
        \item all self-dual uniform matroids;
        \item all identically self-duall uniform matroids;
        \item all self-dual graphic matroids on six or fewer elements;
        \item all identically self-dual graphic matroids on six or fewer elements;
        \item an infinite family of simple graphic self-dual matroids.
    \end{enumerate}
}
\begin{proof}$\,$\pn
    \begin{enumerate}[label=(\roman*)]
        \item 
            Let $U_{n,m}$ be an uniform matroid.\pn
            
            Lets asume that it is self-dual. As
            any basis $B$ for $U_{n,m}$ has size $n$, if it is going to be self-dual,
            then its complement must have size $n$ as well. This means that $m = 2n$.\pn
            
            Now lets suppose that $m = 2n$. So, any subset of size $n$ is a basis for $U_{n,m}$,
            but it is also true that any subset of size $n$ is the complement of another subset
            of size $n$ and then, any basis is the complement of another basis. Then $U_{n,m}$ is
            not only self-dual, but also identically self-dual.
        \item 
            Just as we saw above, if an uniform matroid is self-dual, then it is identically self-dual, and
            an uniform matroid $U_{n,m}$ is self-dual if and only if $m = 2n$.
        \item 
            It is clear that if a matroid is to be self-dual, then its size must be two times
            the size of any of its basis (any forest that is not contained in any other forest). 
            This means that there are only self-dual graphic matroids with an even number of edges. \pn
            
            -2 edges:\pn
            
            There are only 4 posibilities.\pn 
            
            Two disjoint no-loop edges. But two disjoint edges form a forest of size two, but in this case we 
            are loking for forests of size one.\pn
            
            Two loops. Two loops doesn't contain any non empty forest at all, so it will not work.\pn
            
            One no-loop edge and one loop. In this case, we have only a tree of size one. That gives us a basis of size one
            and its complement give us a cobasis of size one. As the tree is a set that is contained in all the basis,
            and it is minimal non empty with such property it is a coloop. Then the dual matroid consits of a cobasis and
            a coloop, both of size one and disjoint. Then it is clear thet there is an isomorphism, one that sends the
            no-loop edge to the only edge in the coloop and the loop to the only edge in the cobasis.\pn
            
            The last posibility which is two parallel edges, has two basis of size one, and it is easy to see that this 
            matroid is not only self-dual but also indentically self-dual.\pn
          
            -4 edges:\pn
            
            Any combination of two of the 2-edges self-dual graphic matroids will do the trick. This gives us
            three possible combinations. And only one of them will be identically self-dual (the one composed of two identically
            self dual matroids).
            
            There is only another one and it is the graph that results from a triangle adding a parallel edge to one of its edges.
            An isomorphism would be sending the simple edges to the parallel edges and the parallel edges to the simple edges.
            The two simple edges form a set that intersects any of the basis of the matroid, so it is a cocircuit. 
            The proposed isomorphism send that cocircuit to a circuit in the original matroid. 
            The two parallel edges along with any of the simple edges form another set that intersects any of the basis of the original
            matroid, so it is another cocircuit in the dual matroid. Those edges are sent to a triangle in the original matroid.
            There are two of these cocircuits and each one is sent two one of the two possible triangles in the original matroid.
            Any basis formed for a simple edge and one of the two parallel edges has as complement the other simple edge and the other
            parallel edge, and under the proposed isomorphism any of such basis are sent to other of the same type.
            The only other basis that is left is the one that consits of the two simple edges, and they are sent to the pair of parallel
            edges, which is independent in the dual matroid. We have proved so far that this isomorphism sends any of the 
            basis in the original matroid to a basis in the dual matroid and any cocircuit to  a circuit in the original matroid. Then such isomorphism
            is a matroid isomorphism between the graphic matroid and its dual.\pn
            
            Lemma 2.3.7 from~\cite{Oxley} stays that if $G^*$ is a geometric dual of the planar graph $G$, then
            $M(G^*) \cong M^*(G)$. In particular for self-dual matroids we will have that $M(G^*) \cong M^*(G) \cong M(G)$.
            
            From now on we will use this fact to look for the remaining self-dual matroids. As the maximum size of these matroids
            is $6$ and $K_4$ is a complete planar graph with 6 edges, then every one of the matroids will have a planar representation. And the
            previously mentioned lemma will let us work with graph representations instead of a complicated structure of subsets and its 
            intersections.\pn
            
            -6 edges:\pn
            MISSING!
        \item 
            Any graph on six or fewer elements obtained from a forest adding a parallel edge to each edge in the forest.
            MISSING!
            
        \item 
            Any wheel.
    \end{enumerate}
\end{proof}