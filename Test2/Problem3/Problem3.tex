\prob
{
    Let $e$ and $f$ be distinct elements of a matroid. Prove that every circuit containing $e$ also contains $f$ if and only if 
    $\{e\}$ or $\{e, f\}$ is a cocircuit.
}
\begin{proof}
    $\,$\pn
    \textbf{Sufficiency}\pn
        Suppose every circuit containing $e$ also contains $f$.\pn 
        
        If every basis contains $e$, then, no cobasis contains $e$, which means $\{e\}$ is a
        cocircuit.\pn
        
        Suppose then that there is a basis $B$ such that $e \notin B$. As $C(e, B)$ is a circuit that contains $e$, then
        $f \in B$. That is, any basis that doesn't contain $e$, contains $f$. That is, any cobasis that contains $e$, doesn't 
        contain $f$.\pn 
        
        Note that $B \setminus \{f\} \cup \{e\}$ is a basis given that it has the right size to be a basis and
        it is obtained from $B \cup \{e\}$ getting rid of the only circuit that it contains (that is, it is independent). Then, there
        is a basis that doesn't contains $e$ and then $\{e\}$ is coindependent, and there is also a basis that doesn't contain $f$ and then
        $\{f\}$ is also coindependent.\pn
        
        Now suppose there is a basis $B'$ such that $f \notin B'$. If $e \notin B'$, then $C(e, B')$ is a circuit that contains
        $e$ but not $f$ whitch contradicts our hyphothesis. Then any basis that doesn't contain $f$, contains $e$. That is,
        any cobasis containg $f$, doesn't contain $e$.\pn
        
        Then $\{e, f\}$ is a subset that is not contained in any cobasis, then it is codependent but any of its subsets is coindependent.
        Then it is a cocircuit as we wished to show.\pn
        
    \textbf{Necessity}\pn
        Suppose $\{e\}$ is a cocircuit. Then, again by Proposition 2.1.6 in~\cite{Oxley}, $E(M) \setminus \{e\}$ is an hyperplane, and then
        $e$ is contained in every basis, then $e$ is in no circuits (by [\ref{t1:p9}]). Then it is true that every circuit containing $e$
        also contains $f$, since there are not circuits containing $e$.\pn
        
        Now suppose that $\{e, f\}$ is a cocircuit. Then, again by Proposition 2.1.6 in~\cite{Oxley}, $H = E(M) \setminus \{e, f\}$ is an
        hyperplane, so any basis must contain $e$ or $f$.\pn
        
        Let $C$ be a circuit such that $e \in C$. Then $C \setminus \{e\}$ is independent. Let $B$ be a basis such that 
        $C \setminus \{e\} \subset B$. As $B$ is basis and doesn't contain $e$, then $f \in B$. Notice that $C(e, B) = C$\pn 
        
        Now, $B^* = E(M) \setminus B$ is a cobasis that contains $e$ but doesn't contain $f$, then $C*(f, B^*) = \{e, f\}$ is the
        only cocircuit in $B^* \cup \{f\}$, that means that $B^* \cup \{f\} \setminus \{e\}$ is a coindependent of maximum size, that is
        a cobasis. Taking its complement we will get a basis, and it is
        \begin{align}
                E(M) \setminus (B^* \cup \{f\} \setminus \{e\}) &= B \cup \{e\} \setminus \{ f \}.
        \end{align}
        
        Which means that $f \in C$.
\end{proof}