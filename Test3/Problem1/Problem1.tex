\prob
{
    Give a real representation for $M(K_4)$ that is totally unimodular.
}
\begin{proof}
    Lets take a incidence matrix for $K_4$, and give it any orientation by replacing exaclty one $1$ by a $-1$ in each column. 
    It is a representation on $GF(3)$, we have already justified it in \ref{t1:p1iii}, and for the same justification, it is a 
    representation on $\R$ as well. Les call $M$ such matrix.\pn
    
    \begin{align}
        M =
            \bordermatrix{
                    &       &       &       &       &       &       \cr
                    &   1   &   1   &   1   &   0   &   0   &   0   \cr
                    &  -1   &   0   &   0   &   1   &   1   &   0   \cr
                    &   0   &  -1   &   0   &  -1   &   0   &   1   \cr
                    &   0   &   0   &  -1   &   0   &  -1   &  -1   
            }    
    \end{align}
    
    
    Lets see that it is totally unimodular.\pn
    
    \begin{itemize}
        \item [case $1 \times 1$] 
            There is nothing to prove. The matrix consists of only $1\times 1$ submatrices with 1, 0, or -1.
        
        \item [case $2 \times 2$] 
            Cases with a column or row of zeros have determinat 0. So lets see the remaining cases.\pn 
            
            There cannot be a case with no zero entries there are no two columns that contains non-zero entries in the same coordinates.\pn
            
            So, there should be at least one zero entrie. Now, if you have a column with only one non-zero entrie with value $r$, and the determinant of
            the resultant submatrix obtained by deleting the row and column of that non-zero entrie is $D$, then the determinant of the original
            matrix will be $\pm r D$. So, in our case, $r \in \{-1, 1\}$ and $D \in \{-1, 0, 1\}$. So any $2 \times 2$ submatrix of $M$ will have 
            determinant $-1, 0$ or $1$.
            
        \item [case $3 \times 3$]
            As we have already proved, if a $3 \times 3$ submatrix $N$ of $M$ contains a column with only zero entries or with only one non-zero entrie, then
            its determinant will be $-1, 0$ or $1$. Then the only case left is when any column contain two non-zero entries.\pn
            
            Which is equivalent to choose three edges of $K_4$ such that all of them are different and any of them is incident with the other two.
            Saying this, it is easy to see that $N$ is the incidence matrix of an oriented triangle, and in \ref{t1:p1iii} we saw that such matrices have
            dependent columns set and then its determinant is always $0$.
            
        \item [case $4 \times 4$]
            Again, the only case left to discuss will be the case in which every column has exactly two non-zero entries. That will be equivalent
            to choose four edges in the underlaying oriented $K_4$, as any tree in $K_4$ has exactly three edges, then any four edges will form
            a dependent set (because it will contain cycles and again thanks to \ref{t1:p1}) and then the determinant in such case will always be zero.
    \end{itemize}
\end{proof}