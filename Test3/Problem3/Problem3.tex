\prob
{
    (Whitney 1932a; Ore 1967, Section 3.3) The graph $H$ is a Whitney dual of the graph $G$ if there is a bijection 
    $\psi : E(G) \rightarrow E(H)$ such that, for all subsets $Y \subset E(G)$,
    \begin{align}
            r(M(H)) - r(M(H)\setminus \psi(Y)) = |Y| - r(M(G)|Y).
    \end{align}
    \begin{enumerate}[label=(\roman*)]
        \item Show that if $H$ is a Whitney dual of G, then G is a Whitney dual of $H$.
        \item Determine the relationship between Whitney duals and geometric and abstract duals.
        \item Prove that a graph is planar if and only if it has a Whitney dual.
    \end{enumerate}
}
\begin{proof}
    Before anything, lets make some observations. Let $G$ be a graph and $H = G^*$, let $\psi : E(G) \rightarrow E(G^*)$ be the identity.
    For any $Y \subset E(G)$, we have that
    \begin{align}
            r(M(G^*)) - r(M(G^*)\setminus \psi(Y))  &= |E(G)| - r(G) - (|E(G^*)\setminus Y| - r(G) + r(M(G)|Y))     \\
                                                    &= |E(G)| - r(G) - (|E(G)| - |Y|) + r(G) - r(M(G)|Y)            \\
                                                    &= - r(G) + |Y| + r(G) - r(M(G)|Y)                              \\
                                                    &= |Y| - r(M(G)|Y).
    \end{align}
    
    So, every dual is a Whitney dual. As the rank conserves all of its properties under any matroid isomorphism, any
    matroid isomporphic to $G^*$ will also be a Whitney dual.\pn
    
    Now, suppose $H$ is a Whitney dual of $G$ under a bijection $\psi : E(G) \rightarrow E(H)$, let $X \subset E(H)$, and 
    define $Y = \psi^{-1} (X)$. By definition of Whitney dual, we have
    
        \begin{align}
                r(M(H)) - r(M(H)\setminus \psi(Y)) &= |Y| - r(M(G)|Y) \\ 
                r(M(H)) - r(M(H)\setminus \psi(\psi^{-1} (X))) &= |\psi^{-1} (X)| - r(M(G)|\psi^{-1} (X))\\
                r(M(H)) - r(M(H)\setminus X) &= |\psi^{-1} (X)| - r(M(G)|\psi^{-1} (X)) \label{this_bitch}\\
        \end{align}
    Setting $X =  E(H)$ we get
    
        \begin{align}
                r(M(H)) = |E(G)| - r(M(G)) \label{this_new_bitch}
        \end{align}


    From \eqref{this_bitch}, making use of $E(H) \setminus X$ instead of $X$ we have
    \begin{align}
            r(M(H)) - r(M(H)\setminus (E(H) \setminus X)) &= |\psi^{-1} (E(H) \setminus X)| - r(M(G)|\psi^{-1} (E(H) \setminus X)) \\
            r(M(H)) - r(M(H)|X) &= |\psi^{-1} (E(H) \setminus X)| - r(M(G)\setminus \psi^{-1} ( X)) \\
            r(M(H)|X) &= r(M(G)\setminus \psi^{-1} ( X)) -|\psi^{-1} (E(H) \setminus X)| + r(M(H)) \label{this_third_bitch}\\
    \end{align}
        
    Finally, from \eqref{this_third_bitch} we have
    \begin{align}
            r(M(H)|X)   &=  r(M(G)\setminus \psi^{-1} (X)) -|\psi^{-1} (E(H) \setminus X)| + r(M(H))                \\
                        &\comment{applying \eqref{this_new_bitch}}                                                  \\
                        &=  r(M(G)\setminus \psi^{-1} (X)) -|\psi^{-1} (E(H) \setminus X)| + |E(G)| - r(M(G))       \\
                        &\comment{reordering and grouping}                                                          \\
                        &=  r(M(G)\setminus \psi^{-1} (X)) - r(M(G))  + (|E(G)| -|\psi^{-1} (E(H) \setminus X)|)    \\
                        &\comment{rewriting $|\psi^{-1} (E(H) \setminus X)|$ using that $\psi$ is bijection}                                                                        \\
                        &=  r(M(G)\setminus \psi^{-1} (X)) - r(M(G))  + (|E(G)| - |E(G)| - |\psi(X)|)               \\
                        &\comment{simplifying}                                                                      \\
                        &=  r(M(G)\setminus \psi^{-1} (X)) - r(M(G))  - |\psi(X)|                                   \\
                        &\comment{applying equivalence of rank function Proposition 1.2.9 \cite{Oxley}}             \\
                        &=  r^*(M(G)|\psi^{-1}(X))
    \end{align}
    
    We have proved that $\psi$ is a bijection such that the rank function $r$ and $r^*$ for $M(H)$ and $M^*(G)$ respectively remain invariant under
    $\psi$, that is, we have proved that $M(H)$ is isomporphic to $M^*(G)$ under the bijection $\psi$.
    
    The rest is easy.\pn
    \begin{enumerate}[label=(\roman*)]
        \item   Is obvious applying that if $M(H)$ is isomporphic to $M^*(G)$ then $M(G)$ is isomporphic to $M^*(H)$.
        \item   Every abstract dual is a Whitneys dual and vice-versa as we have proved above. Then every geometric dual is
                a Whitney dual. But using that the geometric dual is always connected, and the results from above, any
                no connected graph will be Whitneys dual of its own dual, but it can not be its geometric dual, since this last
                is always connected.
        \item   If a graph is planar, then any geometric dual is a Whitney dual. \pn
        
                If a graph has a Whitney dual under a bijection $\psi$, then such Whitney dual is an abastract dual under the
                same bijection $\psi$, and graph is planar if and only if it has an abstract dual.\pn
    \end{enumerate}
\end{proof}
