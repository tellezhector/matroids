\prob
{
    (Lehman 1964) Let $e$ be an element of a connected binary matroid $M$ and let $\mathcal{C}_e$ be the set of circuits
    containing $e$. Show that the circuits of $M$ not containing $e$ are precisely the minimal non-empty sets of the
    form $C_1 \triangle C_2$ where $C_1, C_2 \in \mathcal{C}_e$.
}
\begin{proof}

    We are going to make strong use of the theorem $9.1.2$ from \cite{Oxley}, which among other things states that
    a matroid $M$ is binary if and only if for every pair of distinct circuits $C_1, C_2$, $C_1 \triangle C_2$ is disjoint
    union of circuits.\pn
    
    First we are going to prove that if $C$ is a circuit of $M$ that doesn't contains $e$, then there are $C_i, C_j \in \mathcal{C}_e$ 
    such that $C = C_i \triangle C_j$.\pn
    
    There is always a circuit $C_i$ such that it contains $e$ and it intersects $C$. To see this
    let $f \in C$. As $M$ is connected there is a circuit $C_i$ such that $e, f \in C_i$ and therefore $f \in C \cap C_i$.\pn
    
    Suppose that $C_i$ is such that $C \triangle C_i$ is a circuit, lets call such circuit $C_j$. $C_i$ and $C_j$ are
    both circuits that contain $e$ and the following holds:\pn
    
    \begin{align}
            C_j &= C \triangle C_i   \\
            C_i \triangle C_j &= C \triangle C_i \triangle C_i       \\     
            C_i \triangle C_j &= C.            
    \end{align}
    
    And we would be done. Now lets prove that such circuit $C_i$ exists.\pn
    
    Let $C_i$ be a circuit that intersects $C$ and contains $e$ such that $C_i \setminus C$ is minimal. We claim that $C_i$ is such that
    $C_i \triangle C$ is a single circuit. Suppose it is false, any circuit contained in $C_i \triangle C$ must intersect $C \setminus C_i$, 
    otherwise it will be properly contained in $C_i$, which is a contradiction to circuits minimality, for the same reason it must intersect 
    $C_i \setminus C$ or it will be properly contained in $C$.\pn
    
    There must be a circuit $C_i' \subset C_i \triangle C$ that contains $e$. As at least another circuit intersects $C_i \setminus C$, then
    $C_i' \setminus C \varsubsetneq C_i \setminus C$, contradicting the minimality of $C_i \setminus C$ under the circuits that contain $e$ and
    intersect $C$.\pn
    
    Now we are going to prove that the minimal non-empty sets of the form $C_1 \triangle C_2$ where $C_1, C_2 \in \mathcal{C}_e$ are precisely all the 
    circuits that doesn't contain $e$.\pn
    
    Suppose $C_1 \triangle C_2$ is minimal with such property but it is not a circuit. It is clear that $C_1 \triangle C_2$ doesn't contain $e$ as
    $e \in C_1 \cap C_2$. We know from the theorem $9.1.2$ from \cite{Oxley}  that $C_1 \triangle C_2$ is union of disjoints circuits. Let $C$ be
    any of those circuits, from the previous part we know that we can construct $C_i, C_j \in \mathcal{C}_e$ such that $C = C_i \triangle C_j$, contradicting
    the minimality of $C_1 \triangle C_2$.
\end{proof}