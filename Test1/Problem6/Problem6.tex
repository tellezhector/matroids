\prob
{\label{t1:p6}
    Prove that a matroid $M$ is uniform if and only if it has no circuits of 
    size less than $r(M) + 1$.
}
\begin{proof}
    $\,$\pn
    \textbf{Sufficiency}\pn

	Suppose that $M$ is a uniform matroid $U_{m, n}$, then $r(M) = m$ by definition.
    Moreover, any $A \subset E(M)$ such that $|A| < m$ will be subset of some basis and
    therefore independent. Then if there is any circuit $C$, it must have size greater
    than $m$, that is $r(C) \geq m + 1 = r(M) + 1$.\pn

    \textbf{Necessity}\pn
    
    Suppose now that $M$ is such that it has no circuits of size less than $r(M) + 1$.
    Then, for any $A \subset E(M)$ with $|A| \leq r(M)$ it must be independent, if not,
    then $A$ should contain a circuit, but any circuit has size at least $r(M) + 1$.\pn
    
    Then $M$ must be isomorphic to an uniform matroid $U_{r(M), |E(M)|}$.
\end{proof}