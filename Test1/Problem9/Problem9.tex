\prob
{
    Prove that statments (a)-(g) below are equivalent for an element $e$ of a matroid $M$:
    \begin{enumerate}[label=(\alph*)]
        \item $e$ is in every basis.
        \item $e$ is in no circuits.
        \item If $X \subseteq E(M)$ and $e \in cl(X)$, then $e \in X$.
        \item $r(E(M) \setminus \{e\}) = r(E(M)) - 1$.
        \item $E(M) \setminus \{e\}$ is a flat.
        \item $E(M) \setminus \{e\}$ is a hyperplane.
        \item If $I$ is an independent set, then so is $I \cup \{ e \}$.
    \end{enumerate}
}
\begin{proof}
    $\,$\pn
    \textbf{[(a) $\Rightarrow$ (b)]} \pn
        Suppose there is a circuit $C$ such that $e \in C$, then $C \setminus \{ e \} \in \I(M)$, then
        there must be $B \in \B$ such that $C \setminus \{ e \} \subseteq B$, but $e \not\in B$, otherwise
        $C \subset B$. But this contradicts that $e$ is in every basis and therefor there is not such 
        circuit $C$.\pn
        
    \textbf{[(b) $\Rightarrow$ (c)]} \pn
        Let $X \subseteq E(M)$ such that $e \in cl(X)$. Suppose that $e \notin X$ and let $B_X$ be a basis of $X$.
        we have that $x \in cl(X) = cl(B_X)$, then $cl(B_X) = cl(B_X \cup \{ e \})$, which means that
        $r(B_X \cup \{ e \}) = r(B_X) = |B_X| < |B_X \cup \{ e \}|$, and therefore $B_X \cup \{ e \}$ is a dependent set,
        and it must contain a circuit $C$, as $C$ can not be subset of $B_X$ (because $B_X$ is independent) then $e \in C$,
        which is a contradiction to (b).\pn

    
    \textbf{[(c) $\Rightarrow$ (d)]} \pn
        As $E(M) \setminus \{e\}$ is missing only one element of $E(M)$, then $r(E(M) \setminus \{e\}) \geq r(E(M)) - 1$.
        If $r(E(M) \setminus \{e\}) = r(E(M))$ then $cl(E(M) \setminus \{e\}) = E(M)$ and $e \in E(M)$ but $e \notin E(M) \setminus \{e\}$,
        which contradicts (c).\pn
        
    \textbf{[(d) $\Rightarrow$ (e)]} \pn
        By definition of closure, $E(M) \setminus \{ e \} \subseteq cl(E(M) \setminus \{ e \})$. And $E(M) \setminus \{ e \}$ is contained in only
        two subsets of $E(M)$, there are $E(M)$ itself and $E(M) \setminus \{ e \}$. So $cl(E(M) \setminus \{ e \}) = E(M) \setminus \{ e \}$ or
        $cl(E(M) \setminus \{ e \}) = E(M)$. If the second occurs, then $r(E(M) \setminus \{ e \}) = r(E(M))$, which contradicts (d), and therefore
        $cl(E(M) \setminus \{ e \}) = E(M) \setminus \{ e \}$, which means that $E(M) \setminus \{ e \}$ is flat.\pn
        
    \textbf{[(e) $\Rightarrow$ (f)]} \pn
        As $E(M) \setminus \{ e \}$ is flat, then $r(E(M) \setminus \{ e \}) < r(M)$, but $E(M) \setminus \{ e \}$ is missing only $e$, so 
        $r(E(M) \setminus \{ e \}) = r(M) - 1$ and then $E(M) \setminus \{ e \}$ is an hyperplane.\pn
        
    \textbf{[(f) $\Rightarrow$ (g)]} \pn
        Let $I \in \I(M)$ such that $e \notin I$ and let $B \in \I$ be a basis for $E(M) \setminus \{ e \}$ containing $I$. 
        As $|B| = r(B) = r(E(M) \setminus \{ e \}) = r(M) - 1$, then $r(B \cup \{ e \}) = r(M) = |B| + |\{ e \}| = |B \cup \{ e \}|$ and therefore
        $B \cup \{e\}$ is a basis, and $I \cup \{ e \} \subseteq B \cup \{ e \}$ so $I \cup \{e\}$ is independent.

    \textbf{[(g) $\Rightarrow$ (a)]} \pn
        Let $B \in \B(M)$, as $B \cup \{ e \} \in \I(M)$ and $B$ is maximal by contention over independent sets, so there must be the case that
        $B \cup  \{ e \} = B$.        
\end{proof}