\prob
{
    Characterize the circuits of $M[\c{A}]$ in terms of the bipartite graph $G = \Delta[\c{A}]$.
}
\begin{proof}
    Lets remember that $G = \Delta[\c{A}]$ is a bipartite graph with partition $(S, \c{A})$ and that
    in terms of transversal matroids, a subset $T \subset S$ is independent if there is an injection
    $\phi : T \rightarrow \c{A}$ such that $(t, \phi(t))$ is and edge for every $t \in T$.\pn
    
    Halls theorem states that a susbset $T \subset S$ is independent if and only if $|U| \leq |N_G(U)|$.
    For every $U \subset T$.\pn
    
    That is, a subset $T \subset S$ is dependent if and only if there is $U \subset T$ such that $|U| > |N_G(U)|$. 
    We are looking for subsets $T$ that satisfies this condition and that are minimal.\pn
    
    If there is $U \subsetneq T$ such that  $|U| > |N_G(U)|$, then $T$ can not be minimal.
    
    Then $T$ must has the property $|T| > |N_G(T)|$ (we will refer to this property as the
    $(*)$-property) and its minimal.\pn

    Then, $T$ is a circuit if and only if it is has the $(*)$-property and its minimal with such condition.
    
    We can learn a little more about these $T$'s. Lets check what happen if the graph induced by $T$ has at least 
    $2$ connected components. Lets call $T_1$ and $T_2$ such that $T_1$ induces one connected component and $T_2 = T \setminus T_1$.
    Notice that $N_G(T_1) \cap N_G(T_2) = \emptyset$, and then $|N_G(T)| = |N_G(T_1)| + |N_G(T_2)|$. 
    As $|T| = |T_1| + |T_2| > |N_G(T_1) | + |N_G(T_2)| = |N_G(T)|$ 
    it must be that either $|T_1| > |N_G(T_1)|$ or $|T_2| > |N_G(T_2)|$, without loss of generality, 
    if the first one is true, then at least $T_1$ is smaller than $T$ and then $T$ is not minimal. 
    Then the graph induced by $T$ must     be connected.\pn
     
    %$P$ is switching from $T$ to $\c{A}$. $P$ starts in $\c{A}$, then you can 
    %enumerate the edges of $P$ starting from a start edge and giving the next number to the only edge that is connected to the already numbered
    %edges, if you take the odd edges, as $P$ is spaning, then you can cover all of $T$, which is a contradiction because $T$ had
    %the $(*)$-property. Then $P$ has not start vertex in $\c{A}$, with the same procedure we can show that $P$ has no end vertex in $\c{A}$.\pn
    %
    %Then any $P$ must start and end in $T$, and then, $P$ is even. Notice then, that $|T| = |N_G(T)| + 1$. 
    %If you remove the start (or end) vertex $v$, you get a subpath that starts (ends) in $\c{A}$ and then it covers all of $T \setminus \{v\}$ with 
    %a matching as seen above. If you remove any other vertex $v$ from $T$ different from the start and end vertices, then you make two paths that 
    %start (end) in $\c{A}$ and then $T \setminus \{v\}$ is covered by a matching. That is, if you remove any vertex, you can have a matching that 
    %covers the vertices from $T$ left, that is,if you remove any vertex the $(*)$-property is loss \pn 
    %
    %Then we can characterize the circuits of $M[\c{A}]$
\end{proof}